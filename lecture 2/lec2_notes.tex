\documentclass{article}

\title{\underline{Non-personalised collaborative filtering (CF) \& user-based CF}}
\author{Dom Jina}
\date{\today}

\begin{document}

\maketitle

\newpage

\underline{\textbf{non-personalised collaborative filtering}} \newline

\textbf{Collaborative filter} - tell me whats popular among peers \\

User ratings and actions are noisy measurements

Relevance is a user's perogative \newline

\underline{\textbf{Feedback model}} \\

\underline{Explicit feedback} - Reviews, likes, comments, shares

If not a lot of rating data, a lot of noise i.e. 8/10 for 9 reviews $>$ 7/10 for 10,000 reviews\\

\textbf{cons}: are ratings reliable and accurate \\

\underline{implicit feedback} - abundant data from user actions - views, clicks, reads, buys, etc \\

Not direct expressions of preferences

Didn't click: bad or didn't see?

\textbf{cons}: lots of storage needed

\newpage

\underline{\textbf{collaborative filtering}} \newline

Leveage "wisdom of the crowds"

Taking the results of others to predict the values of an unknown.

\underline{Problem}: will be the same for all users, however not all users will like the same thing

Could try segment users, i.e. age groups, income, location, etc

Not fully personalised. \\

Use sessions and associtions

Use historical profiles - may introduce spurious associtions

Use transaction data - may limit follow up sales

Time constrained profiles - offer a compromise \\

\underline{Association rule mining} \\
Given a set of transactions, find rules that will predict the occurence of an item based on occurence of others.

Look at frequent items sets and partition them

Rules are not from one item to another

They are from one set to another i.e. one or more than one (set)\\

item set - A colleciton of one or more items, a k-item set is exactly k items \\
support count is frequency of occurence of an itemset\\
support fraction of all transactions that contain an itemset\\
Association rule between one set and another\\
Rule evaluation metrics fraction of transactions that contain all items of X and Y\\
Confidence of an item measures how often transactions containing Y appear within the transactions that contain X\\

\end{document}